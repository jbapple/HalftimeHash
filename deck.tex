\documentclass[xcolor=dvipsnames]{beamer}
%\usecolortheme[named=Black]{structure}
%\PassOptionsToPackage{hyphens}{url}\usepackage[pdfusetitle,hidelinks]{hyperref}

\AtBeginDocument{%\bfseries
  \Large\selectfont} % normal text 

%% \setbeamercolor{palette primary}{bg=black,fg=white}
%% \setbeamercolor{palette secondary}{bg=black,fg=white}
%% \setbeamercolor{palette tertiary}{bg=black,fg=white}
%% \setbeamercolor{palette quaternary}{bg=black,fg=white}

%% \setbeamercolor{section in toc}{bg=black,fg=black} % TOC sections

%\setbeamercolor{structure}{bg=black,fg=white} % itemize, enumerate, etc
%\setbeamercolor{normal text}{bg=black,fg=white} % TOC sections

%\setbeamerfont{normal text}{series=\bfseries,size=\Huge}
\setbeamerfont{title}{size=\Huge}
\setbeamerfont{frametitle}{size=\Huge}

\newcommand{\defeq}{\;\genfrac{}{}{0pt}{2}{\text{def}}{=}\;}
\DeclareMathOperator{\adj}{adj}
%\DeclareMathOperator{\Pr}{Pr}

\usepackage{verbatim}
\usepackage{amsfonts}
\usepackage{hyperref}

\renewcommand*{\thefootnote}{\fnsymbol{footnote}}

\begin{document}

\title{HalftimeHash: Modern Hashing without 64-bit Multipliers or Finite Fields}
\author{Jim Apple}
\date{}
\institute{\href{mailto:jbapple@apache.org}{\Large jbapple@apache.org}}

\frame{\titlepage}

\begin{frame}
  HalftimeHash is a new hash function designed for long input (1KB+).
\end{frame}

%% \begin{frame}
%%   \begin{center}
%%     \includegraphics[width=9.0cm]{line-cl-hh24-floor.eps}
%%   \end{center}
%% \end{frame}

%% \begin{frame}
%%   \frametitle{Why a hash function for long strings?}
%%   Universe collapsing hashes long strings into short strings (16 bytes to 40 bytes)
%%   % After collapsing the universe, use hash composition:
%%   % Use tabulation or SipHash on the resulting short string
%%   \pause \\
%%   $ $ \\
%%   Useful for randomized data structures where randomness depends on key hashes, such as hash tables or Bloom filters
%% \end{frame}

\begin{frame}
  \frametitle{Why not use 64-bit multipliers or finite fields?}
  Multiplying two 64-bit numbers to produce a 128-bit number is expensive % and is required for fields with more than $2^{32}$ elements.
  \pause

  $ $\\
  It is not present in all common ISAs
  \pause

  $ $\\
  It is missing from some popular programming languages, including Java, Python, and Swift
\end{frame}

\begin{frame}
  \frametitle{But \ldots}
  32-bit hash functions produce much higher collision probability: $2^{-32}$, not $2^{-64}$
\end{frame}

\begin{frame}
  Overcoming that limitation is the main contribution of this work
\end{frame}

\begin{frame}
  \begin{center}
    \includegraphics[width=9.0cm]{amd-cl-hh24-floor.eps}
  \end{center}
\end{frame}



%% \begin{frame}
%%   \frametitle{Why tunable collision probability?}

%%   In HalftimeHash, the longer the output length, the lower the collision probability:
%%   \pause

%% %Roughly $(2^{-32}\lg n)^{(b-8)/4}$
%% %\pause
%% $ $ \\ % For 250kb
%%  $\varepsilon = 2^{-59}$ for $16$-byte output %, 89, 116,
%% and $2^{-145}$ for $40$-byte output

%% \end{frame}


%% \begin{frame}
%%   \frametitle{Why tunable collision probability?}

%%   The basic primitive has 8-byte output and collision probability more than $2^{-32}$.
%%   This is too high for many applications
%%   \pause

%%   $ $ \\
%%   Lower collision probability could be achieved by hashing twice or more
%%   \pause

%%   $ $ \\
%%   Avoiding this extra computation is the theory contribution of this paper
%%   %Engineering discussion comes after.
%% %Roughly $(2^{-32}\lg n)^{(b-8)/4}$
%% %\pause

%% \end{frame}



%% \begin{frame}
%%   \begin{center}
%%     \includegraphics[width=9.0cm]{amd-cl-hh24-floor.eps}
%%   \end{center}
%% \end{frame}


\begin{frame}
  \frametitle{How do the fastest hash functions work?}

  For UMASH and clhash:
  \pause
  $ $\\
  \begin{enumerate}
  \item Divide the input into equal-length constant-sized blocks
    \pause
  \item Hash each block down to 128 bits
    \pause
    \item Combine the results using polynomial hashing
  \end{enumerate}
\end{frame}

\begin{frame}
  \frametitle{How do the fastest hash functions work?}
  Most of the work is done in step two, hashing each block down to 128 bits
  \pause \\ $ $ \\
  In clhash and UMASH, the collision probability at this step is $2^{-64}$.
  \pause \\ $ $ \\
  In HalftimeHash the collision probability can be as low as $2^{-145}$ (for 320-bit codomain)
\end{frame}


%% \begin{frame}
%%   \frametitle{Encode-hash-combine}
%%   Let $F$ be a hash function\footnote{Family of hash functions} from $V^n$ to $V$ with collision probability of $\varepsilon$
%%   \pause

%%   $ $ \\
%%   EHC stretches it to $V^{k+1}$ and $\varepsilon ^ {k+1}$
%%   %% \begin{itemize}
%%   %% \item Almost-universal universe-collapsing family from $V^n$ to $V^{m+1}$ for specific $n$ and $m$
%%   %%   \pause
%%   %% \item Applied at the leaves, this improves $\varepsilon$ to $\varepsilon^{m+1}$ without having to do $m$ tree hashes
%%   %% \end{itemize}
%% \end{frame}

\begin{frame}
  \frametitle{Encode-hash-combine}
%  Collapsing $V^n$ to $V^{k+1}$:
  \begin{enumerate}
  \item Apply an erasure encoding that slightly extends $V^n$ to $V^{n+k}$
    \pause
  \item Hash each component of $V^{n+k}$ independently
    \pause
  \item Apply a linear map from $V^{n+k}$ to $V^{k+1}$
  \end{enumerate}
  \pause
  The output has collision probability $\varepsilon^{k+1}$
\end{frame}

\begin{frame}[fragile]
  \frametitle{Encode-hash-combine}
  This is an old result for $V$ a field
  \pause
  \\
$ $\\
  Not applicable in \verb|uint64_t|
\end{frame}

%% \begin{frame}
%%   \frametitle{Encode-hash-combine}
%%   Each of these (encoding, component hash, linear map) must have certain characteristics
%% \end{frame}

%% \begin{frame}
%%   \frametitle{Encode-hash-combine: encode}
%%   Must have {\em distance} $k+1$: Hamming distance between outputs is never in $[1,k]$
%% \end{frame}

%% \begin{frame}
%%   \frametitle{Encode-hash-combine: hash function from $V$ to $V$}
%%   Must be almost-$\Delta$-universal:
%%   \[
%%   \forall \delta, a \neq b, \Pr_s[H_s(a) - H_s(b) = \delta] \le \varepsilon
%%   \]
%% \end{frame}

\begin{frame}
  \frametitle{Encode-hash-combine: linear map}
  \begin{enumerate}
  \item Apply an erasure encoding that slightly extends $V^n$ to $V^{n+k}$
  \item Hash each component of $V^{n+k}$ independently
  \item Apply a linear map $T$ from $V^{n+k}$ to $V^{k+1}$
  \end{enumerate}
\end{frame}


\begin{frame}
  \frametitle{Encode-hash-combine: linear map}

  \begin{enumerate}
{\color{gray}  \item Apply an erasure encoding that slightly extends $V^n$ to $V^{n+k}$}
{\color{gray}  \item Hash each component of $V^{n+k}$ independently}
  \item Apply a linear map $T$ from $V^{n+k}$ to $V^{k+1}$
  \end{enumerate}
$ $ \\
  Let $T'$ be the set of all $k+1$ combinations of columns of $T$

  \pause $ $ \\
  Must have $\forall S \in T', S$ is non-singular
\end{frame}

\begin{frame}

  \frametitle{Encode-hash-combine: linear map \textbf{ in $\mathbb{Z}_{2^{64}}$}}

  Still must have $\forall S \in T', S$ is non-singular in $\mathbb{Z}$, but the conclusion is weaker.
  \pause

  $ $\\
  Let $\widehat{x}$ denote the largest power of $2$ that divides $x$.
  \pause

  $ $\\
  The smaller $\widehat{x}$ is, the closer $x$ is to invertible in $\mathbb{Z}_{2^{64}}$:
  \[
  \widehat{x} = \left|\{y : xy = 0\}\right| + 1
  \]
\end{frame}

\begin{frame}

  \frametitle{Encode-hash-combine: linear map \textbf{ in $\mathbb{Z}_{2^{64}}$}}

  Let $m$ be $\max_{S \in T'} \widehat{\det(S)}$.
  \pause

  $ $ \\
  Then the collision probability of EHC in $\mathbb{Z}_{2^{64}}$ is not $\varepsilon^{k+1}$, but $(m \varepsilon)^{k+1}$
\end{frame}


\begin{frame}

  \frametitle{Engineering HalftimeHash}
  \pause
  \begin{itemize}
  \item Use a code that can be evaluated without any multiplications
  \pause
  \item Use $T$ such that most scalars are powers of two, for shifting
  %% \pause
  %% \item Use NH for hashing medium-length input into short output, a family that takes half as many multiplications as its input size
  \pause
  \item Supplement EHC with a Merkle-tree-like construction without requiring finite fields
  \pause
  \item Use AVX2 and AVX-512 instructions to perform 4-8 multiplications simultaneously
  \end{itemize}
\end{frame}

\begin{frame}
  \frametitle{Engineering HalftimeHash}
  \begin{center}
    \includegraphics[width=8.0cm]{line-cl-hh24-floor.eps}
  \end{center}
\end{frame}

\begin{frame}
  \frametitle{Engineering HalftimeHash}
  \begin{center}
    \includegraphics[width=8.0cm]{amd-cl-hh24-floor.eps}
  \end{center}
\end{frame}

\begin{frame}
  \frametitle{Engineering HalftimeHash}
  \begin{center}
    \includegraphics[width=9.0cm]{speed-v-epsilon.eps}
  \end{center}
\end{frame}


\begin{frame}
  \begin{center}
    \href{https://github.com/jbapple/HalftimeHash}{github.com/jbapple/HalftimeHash}
    $ $ \\
    $ $ \\
    \href{mailto:jbapple@apache.org}{\Large jbapple@apache.org}
  \end{center}
\end{frame}

\begin{frame}
  \frametitle{Encode-hash-combine: linear map \textbf{ in $\mathbb{Z}_{2^{64}}$}}

  If two input strings differ, their Hamming distance after encoding is at least $k+1$.
  Let $x$ and $y$ be their encoded values.
  \pause

  $ $\\
  We want to show that $\Pr_s\left[T(H_s(x)-H_s(y)) = \delta\right] \le (m \varepsilon)^{k+1}$
  \pause

  $ $\\
  Let $U$ be a set of $k+1$ places where $x$ and $y$ differ.
%  If The probability that $H_s(x) = H_s(y)$ is small: $\varepsilon^k$.
%  Otherwise, there is at least one difference.
  Let $R$ be the subset of $T$'s columns indexed by $U$
\end{frame}

\begin{frame}
  \frametitle{Encode-hash-combine: linear map \textbf{ in $\mathbb{Z}_{2^{64}}$}}

  The adjugate of $R$ has entries in $\mathbb{Z}$, and so can be interpreted over $\mathbb{Z}_{2^{64}}$.
  \pause

  $ $\\
  $R$ is also non-singular in $\mathbb{Z}$, so $R^{-1}$ exists in $\mathbb{Q}$ and $R^{-1} = \adj(R) / \det(R)$.
\end{frame}

\begin{frame}
  \frametitle{Encode-hash-combine: linear map \textbf{in $\mathbb{Z}_{2^{64}}$}}

  $ $\\
  Let $\det(R) = o 2^{m'}$, where $o$ is odd. Note that $2^{m'} \le m$ and $o$ has an inverse in $\mathbb{Z}_{2^{64}}$.

\end{frame}

\begin{frame}
  \frametitle{Encode-hash-combine: linear map \textbf{in $\mathbb{Z}_{2^{64}}$}}

  After some arithmetic manipulation, can show that
  \[
  \begin{array}{lrcrcl}
    \Pr_s[&R(H_s(x) - H_s(y)) & \equiv & \delta \bmod 2^{64}] & \le & \\
    % adjugate sends equal to equal, but might also make non-equal, equal
    \Pr_s[&2^{m'}(H_s(x) - H_s(y)) & \equiv & o^{-1} \delta \bmod 2^{64}] & = & \\
    \Pr_s[&2^{m'}(H_s(x) - H_s(y)) & \equiv & 2^{m'} \delta' \bmod 2^{64}] & = & \\
    % Another way to think og this next step: the high m' bits of \delta' can be anything
    \Pr_s[&\bigvee_{0 \le c < 2^{m'}} H_s(x) - H_s(y) & = &  \delta'  + c 2^{64 -m'}] &  & \\
    % union bound
    & & &\le  (2^{m'} \varepsilon)^{k}
  \end{array}
  \]
\end{frame}

\begin{frame}
  \frametitle{Engineering HalftimeHash: tree hashing}
  Fast almost-universal families like clhash, UMASH, and Poly1305 use Horner's method to compute hash of arbitrary length strings.
  \pause

  $ $\\
  This, too, uses finite fields.
\end{frame}

\begin{frame}
  \frametitle{Engineering HalftimeHash: tree hashing}
  If $H$ is an almost-universal hash family from $V^2$ to $V$, then
  $G_{s,t} \defeq H_t(H_s(a, b), H_s(c,d))$ is $2 \varepsilon$-almost universal
  \pause
  
  $ $\\
  Iterate to get a family from  $V^{2^i}$ to $V$ that is $i \varepsilon$-almost universal
  % Notice: H(x) is used twice
  % If iterated, this construction tield s universe-collapsing almost-universal hash family
\end{frame}

%% \begin{frame}
%%   \frametitle{Tree hashing}
%%   \begin{itemize}
%%   \item If $H$ is an almost-universal hash function from $V^2$ to $V$
%%     \pause
%%   \item $\forall x, \Pr[H_x(a, b) = H_x(a', b')] < \varepsilon$
%%     \pause
%%   \item $G_{x,y} \defeq H_y(H_x(a, b), H_x(c,d))$ is also almost-universal
%%     % Notice: H_x is used twice
%%   \end{itemize}
%% \end{frame}

\begin{frame}
  \frametitle{Engineering HalftimeHash: tree hashing}
  \begin{center}
    \includegraphics[width=9.0cm]{tree.eps}
  \end{center}
\end{frame}

\begin{frame}
  \frametitle{Engineering HalftimeHash: tree hashing}
  \begin{center}
    \includegraphics[width=9.0cm]{tree-1.eps}
  \end{center}
\end{frame}

\begin{frame}
  \frametitle{Engineering HalftimeHash: tree hashing}
  \begin{center}
    \includegraphics[width=9.0cm]{tree-2.eps}
  \end{center}
\end{frame}

%% \begin{frame}
%%   \frametitle{Engineering HalftimeHash: tree hashing}
%%   \begin{center}
%%     \includegraphics[width=9.0cm]{tree-3.eps}
%%   \end{center}
%% \end{frame}




%% \begin{frame}

%%   \frametitle{Engineering HalftimeHash}
%%   \pause
%%   \begin{itemize}
%%   \item Use a code that can be evaluated without any multiplications
%%   \item Use $T$ such that most scalars are powers of two, for shifting
%%   \item Use NH for hashing, a family that takes half as many multiplications as its input size
%%   \item Supplement EHC with a Merkle-tree-like construction without requiring finite fields
%%   \item Use AVX2 and AVX-512 instructions to perform 4-8 multiplications simultaneously
%%   \end{itemize}
%% \end{frame}


%  Where $\beta \defeq o^{-1}\adj(R)\delta$.
  %Since $R$ is non-singular in $\mathbb{Z}$, the {\em adjoint} of $R$ is $\det(R) \cdot R$, which is also a linear transform over $\mathbb{Z}_{2^{64}}$.

%  \pause
%  $ $\\
%  Let $\det(R) = o 2^{m'}$, where $o$ is odd. $m' \le m$.
% \end{frame}

%% \begin{frame}
%%   Intuition of the proof is that if $a \neq b$, then there are $m+1$ places they likely differ after application of coding and hashing.
%%   \pause
%%   Those places select a non-singular matrix out of the linear transformation.
%%   \pause
%%   Since this matrix has an inverse can apply it to uniquely solve $T \cdot \beta = \Delta$.
%% \end{frame}

%% \begin{frame}
%%   To get a hash value in $V^{m+1}$ from an input string:
%%   \begin{enumerate}
%%   \item Apply EHC from $V^k$ to $V^{m+1}$ on the input, blocked into groups of size $k$.
%%   \item Run $m+1$ copies of tree hash on the result.
%%   \end{enumerate}
%% \end{frame}


%% \begin{frame}
%%   EHC engineering challenges:
%%   \begin{enumerate}
%%   \item Field-based erasure codes are expensive to calculate
%%   \item Application of a linear map requires field multiplications, which are also hard to calculate
%%   \end{enumerate}
%% \end{frame}

%% \begin{frame}
%%   \frametitle{Generalized EHC}
%%   A weakening of EHC that works over $\mathbb{Z}_{2^{64}}$.
%%   \pause \\
%%   Let $p$ be the largest power of 2 that divides the determinant of any
%%   $m+1$ columns in the matrix of the linear map.
%%   \pause \\
%%   Then EHC is $\varepsilon^{m+1}2^{p(m+1)}$-almost universal
%% \end{frame}

%% \begin{frame}
%%   Proof intuition is the same as EHC, with one addition:
%%   \\
%%   In $\mathbb{Z}_{2^{64}}$, some matrices are {\em almost} non-singular, in that there are only a few non-zero vectors that have image $(0, 0)$.
%% \\
%%   For example, the matrix $(1,2),(0,2)$ has only $(0, 2)$ as a non-zero member of the kernel in $\mathbb{Z}_4$.
%%   \\
%% \end{frame}


%% %% \begin{frame}
%% %%   $\varepsilon$-almost universal hash families avoid collisions

%% %%   \[
%% %%   \forall x \neq y \in D,
%% %%   \Pr[H(x) = H(y)] \leq \varepsilon
%% %%   \]

%% %%   %with the probability is taken over the space of the initialization
%% %% \end{frame}

%% \begin{frame}
%%   HalftimeHash building blocks:
%%   \begin{itemize}
%%   \item NH % ($\sum (x_i + k_i) * (x_{i+1} + k_{i+1})$)
%%   \item Erasure codes over $\mathbb{Z}_{2^{64}}^3$
%% %  \item AVX2 and AVX-512 SIMD instructions
%%   \item Tree hashing
%%   \item Encode-Hash-Combine
%%   \end{itemize}
%%   % I'll only cover the last two, as they explain most of the structure of HalftimeHash
%% \end{frame}




%% \begin{frame}
%%   The engineering choices include:
%%   \\
%%   Using non-linear erasure codes over $\mathbb{Z}_{2^{64}}^3$ that only uses Xor's, not multiplication.
%%   \\
%%   Using AVX2 and AVX-512 SIMD instructions that can do up to 16 different $32$-bit $\times 32$-bit $\to 64$-bit multiplications in a single instruction
%%   \\
%%   Terminating the tree hash portion and moving to a different scheme when close to the end of the string
%%   \\
%%   NH hashing: $\sum (x_i + k_i) * (x_{i+1} + k_{i+1})$
%% \end{frame}

%% %% \begin{frame}
%% %%   \begin{center}
%% %%     \includegraphics[width=9.0cm]{smhasher-speed.eps}
%% %%   \end{center}
%% %% \end{frame}

\end{document}

%%  LocalWords:  HalftimeHash treaps EHC
